\begin{topic}{Application Layer}

The Application Layer provides the interface between the end-user and network communication.

Implementation aspects of network protocols
\begin{itemize}
	\item transport-layer service models
	\item client-server paradigm
\end{itemize}

\end{topic}

\begin{topic}{Network Applications}

Network applications run on \texttt{different end systems} (network edges) and \texttt{communicate over the network.}

Network applications \textbf{do not} run on network cores.

Network applications allow for \texttt{rapid app development and propagation.}

\end{topic}

\begin{topic}{Network Architectures}

\begin{itemize}
	\item Client-server
	\item Peer-to-peer (P2P)
\end{itemize}

\begin{subtopic}{2}
\texttt{Client-server Architecture} is the classical architecture consisting of communication between \texttt{multiple clients} and a \texttt{singular server.}

The server is \texttt{always-on} with a \texttt{fixed address} that \texttt{can be scaled} to multiple devices.

Clients communicate with directly with the server and \texttt{do not need to be always on or have a fixed address.} Clients \texttt{do not communicate with each other.}
\end{subtopic}

\begin{subtopic}{3}
\texttt{Peer-to-peer Architecture} is a form of network communication where clients (now peers) do not connect to an always-on server and instead \texttt{communicate directly with each other.}

Peers request service from other peer and provide service in return to other peers. Think torrents.

Peers are \texttt{intermittently connected and can change addresses.}
\end{subtopic}

\end{topic}

\begin{topic}{Processes}

A \texttt{Process} is a program running within a host.

Inter-process communication is two processes communicating on the same host.

Messages are exchanged by processes communicating on different hosts.

\begin{subtopic}{2-}
\textbf{Client process:} initiates communcation

\textbf{Server process:} waits for communcation from clients
\end{subtopic}

\begin{subtopic}{3-}
\texttt{P2P Applications have both client and server processes}
\end{subtopic}

\end{topic}

\begin{topic}{Sockets}

Processes send and receive messages to and from sockets.

\texttt{Sockets} are connections between host devices.

\end{topic}

\begin{topic}{Addressing Processes}

Processes require \texttt{identifiers} so that messages can be sent back to the correct process.

Each \texttt{host} has a \texttt{32-bit IP address.}

A host can have \texttt{multiple processes} so IP addresses are combined with \texttt{port numbers} as \texttt{identifiers.}

\end{topic}