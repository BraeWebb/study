\begin{topic}{Application Layer}

The Application Layer provides the interface between the end-user and network communication.

Implementation aspects of network protocols
\begin{itemize}
	\item transport-layer service models
	\item client-server paradigm
\end{itemize}

\end{topic}

\begin{topic}{Network Applications}

Network applications run on \textbf{different end systems} (network edges) and \textbf{communicate over the network.}

Network applications \textbf{do not} run on network cores.

Network applications allow for \textbf{rapid app development and propagation.}

\end{topic}

\begin{topic}{Network Architectures}

\begin{itemize}
	\item Client-server
	\item Peer-to-peer (P2P)
\end{itemize}

\begin{subtopic}{2}
\textbf{Client-server Architecture} is the classical architecture consisting of communication between \textbf{multiple clients} and a \textbf{singular server.}

The server is \textbf{always-on} with a \textbf{fixed address} that \textbf{can be scaled} to multiple devices.

Clients communicate with directly with the server and \textbf{do not need to be always on or have a fixed address.} Clients \textbf{do not communicate with each other.}
\end{subtopic}

\begin{subtopic}{3}
\textbf{Peer-to-peer Architecture} is a form of network communication where clients (now peers) do not connect to an always-on server and instead \textbf{communicate directly with each other.}

Peers request service from other peer and provide service in return to other peers. Think torrents.

Peers are \textbf{intermittently connected and can change addresses.}
\end{subtopic}

\end{topic}

\begin{topic}{Processes}

A \textbf{Process} is a program running within a host.

Inter-process communication is two processes communicating on the same host.

Messages are exchanged by processes communicating on different hosts.

\begin{subtopic}{2-}
\textbf{Client process:} initiates communcation

\textbf{Server process:} waits for communcation from clients
\end{subtopic}

\begin{subtopic}{3-}
\textbf{P2P Applications have both client and server processes}
\end{subtopic}

\end{topic}

\begin{topic}{Sockets}

Processes send and receive messages to and from sockets.

\textbf{Sockets} are connections between host devices.

\end{topic}

\begin{topic}{Addressing Processes}

Processes require \textbf{identifiers} so that messages can be sent back to the correct process.

Each \textbf{host} has a \textbf{32-bit IP address.}

A host can have \textbf{multiple processes} so IP addresses are combined with \textbf{port numbers} as \textbf{identifiers.}

\end{topic}

\begin{topic}{App-Layer Protocol}

App-Layer Protocol defines:
\begin{itemize}
	\item \textbf{type of message} e.g. request, response
	\item \textbf{message syntax:} message fields and encoding
	\item \textbf{message semantics:} meaning of the fields
	\item \textbf{rules:} how processes should send/receive messages
\end{itemize}

\begin{subtopic}{2-}
Open protocols:
\begin{itemize}
	\item defined in \textbf{RFCs}
	\item allows for \textbf{interoperability}
\end{itemize}
\end{subtopic}

\begin{subtopic}{3-}
Proprietary protocols:
\begin{itemize}
	\item normally implemented for a specific proprietary application
\end{itemize}
\end{subtopic}

\end{topic}

\begin{topic}{Transport Service Considerations}

\begin{subtopic}{1-}
\textbf{Data Integrity}
Reliability of data to reach the destination.
Some applications require all data to reach the destination.
\end{subtopic}

\begin{subtopic}{2-}
\textbf{Timing}
Speed transportation takes.
Some applications require fast transportation to work well.
\end{subtopic}

\begin{subtopic}{3-}
\textbf{Throughput}
Amount of data in a transfer.
Some applications require large throughput while others require minimal throughput.
\end{subtopic}

\end{topic}

\begin{topic}{TCP \& UDP}
\begin{multicols}{2}
\textbf{TCP}\\

\begin{itemize}
	\item \textbf{reliable} transport protocol
	\item \textbf{flow control} prevent overwhelming receiver
	\item \textbf{congestion control} prevent overwhelming network
	\item \textbf{no} timing, minimum throughput guarantee, security
	\item \textbf{setup required} connections need to be established
\end{itemize}

\columnbreak
\textbf{UDP}\\

\begin{itemize}
	\item \textbf{unreliable} transport protocol
	\item \textbf{no} flow control, congestion control, timing, throughput guarantee, security, or connection setup
\end{itemize}
\end{multicols}
\end{topic}

\begin{topic}{Secure TCP}

TCP \& UCP connections have \textbf{no encryption.}

\textbf{SSL} connections are encrypted TCP connections.

SSL connections increase \textbf{data integrity} and offer \textbf{end-point authentication.}

SSL is an application layer protocol. Applications use SSL libraries.

\end{topic}